Данная выпускная квалификационная работа посвящена построение и исследованию эффективности операций в разных формах эллиптических кривых с использованием языков программирования Python и SageMath. Внимание уделено анализу количества операций, необходимых для вычисления кратных точек, и изучению влияния эндоморфизма на эффективность этих алгоритмов.

В рамках работы построена эллиптическая кривая с комплексным умножением, реализованы алгоритмы поиска кратной точки в 4 разных формах эллиптических кривых, подсчитаны количество операций на каждой форме кривой и проведен анализ.

На основе проведенных исследований стало ясно, что наиболее эффективной реализацией среди представленных форм является форма Монтгомери. Эта форма обладает минимальным количеством операций сложения и умножения под XZ координаты, что делает ее привлекательной для использования в цифровой подписи и шифровании с обменом ключей.
Для вычисления y-координаты предпочтительной является форма Скрученного Эдвардса. Форма Хадано с эндоморфизмом эффективнее формы Хадано без эндоморфизма в количестве сложений и умножений, но есть более эффективные формы. 

В дальнейшем планируется развитие исследования с целью поиска и оптимизации других методов вычисления кратной точки, а также сравнительный анализ эффективности различных форм эллиптических кривых и их эндоморфизмов. Это позволит способствовать повышению производительности практических криптографических систем.
