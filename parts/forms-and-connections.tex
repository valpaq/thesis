
Для анализа и сравнения было выбрано 4 формы эллиптических кривых.

\begin{table}[h]
  \centering
  \begin{tabular}{|l|l|l|}
  % \hline
  % $D_{g_{2}, g_{3}}$ & $y^{2} \equiv 4 x^{3}-g_{2} x-g_{3}( \bmod p)$ & дифференциальная форма Вейерштрасса \\
  \hline
  $W_{a, b}$ & $y_{w}^{2} \equiv x_{w}^{3}+a x_{w}+b(\bmod  p)$ & короткая форма Вейерштрасса \\
  \hline
  $M_{\mu, \gamma}$ & $\gamma y_{m}^{2} \equiv x_{m}^{3}+\lambda x_{m}^{2}+x_{m}(\bmod  p)$ & форма Монтгомери \\
  \hline
  $E_{e, d}$ & $e x_{e}^{2}+y_{e}^{2} \equiv 1+d x_{e}^{2} y_{e}^{2}(\bmod  p)$ & искривленная форма Эдвардса \\
  \hline
  $E_{a, b}$ & $ y_{m}^{2} \equiv x_{m}^{3}+a x_{m}^{2}+b x_{m}(\bmod  p)$ & форма Хадано \\
  \hline
  \end{tabular}
  \caption{Исследуемые формы эллиптических кривых}
\end{table}


\subsection{Короткая форма Вейерштрасса}

Кривая, в короткой форме Вейерштрасса~\cite{lang-elliptic}

$$
W_{a, b}: \quad y_{w}^{2} \equiv x_{w}^{3}+a x_{w}+b \quad(\bmod p)
$$

(a) Переход из короткой формы Вейерштрасса в форму Монтгомери задается следующими соотношениями. Пусть $\theta$ - корень многочлена $x^{3}+a x+b$ в поле $\mathbb{F}_{p}$ и элемент $\gamma$ удовлетворяет равенству $\gamma^{2} \equiv 3 \theta^{2}+a(\bmod p)$ и $\gamma \not \equiv 0(\bmod p)$, тогда определим

$$
\mu \equiv \frac{3 \theta}{\gamma} \quad(\bmod p)
$$

Одновременно, любая аффинная точка преобразуется следующим образом:

$$
\left(x_{w}, y_{w}\right) \rightarrow\left(x_{m} \equiv \frac{x_{w}-\theta}{\gamma} \quad(\bmod p), \quad y_{m} \equiv \frac{y_{w}}{\gamma^{2}} \quad(\bmod p)\right)
$$
% \footnotetext{${ }^{2}$ Из (7), (8) следует, что выполнены сравнения $\frac{y_{w}^{2}}{\gamma^{3}} \equiv \gamma y_{m}^{2} \equiv \frac{\left(x_{w}-\theta\right)^{3}}{\gamma^{3}}+\frac{3 \theta\left(x_{w}-\theta\right)^{2}}{\gamma^{3}}+\frac{\left(x_{w}-\theta\right)\left(3 \theta^{2}+a\right)}{\gamma^{3}}$. Домножая на $\gamma^{3}$, раскрывая скобки и учитывая, что $\theta^{3}+a \theta+b \equiv 0(\bmod p)$, получим сравнение, определяющее короткую форму Вейерштрасса $y_{w}^{2} \equiv x_{w}^{3}+a x_{w}-\theta^{3}-a \theta \equiv x_{w}^{3}+a x_{w}+b(\bmod p)$.
% }

Групповой закон в аффинных координатах

Сложение точек:
\begin{align*}
x_3 &= \frac{(y_2 - y_1)^2}{(x_2 - x_1)^2} - x_1 - x_2 \\
y_3 &= \frac{(2x_1 + x_2)(y_2 - y_1)}{x_2 - x_1} - \frac{(y_2 - y_1)^3}{(x_2 - x_1)^3} - y_1
\end{align*}

Удвоение точки:
\begin{align*}
x_3 &= \frac{(3x_1^2 + a)^2}{(2y_1)^2} - x_1 - x_1 \\
y_3 &= \frac{(2x_1 + x_1)(3x_1^2 + a)}{2y_1} - \frac{(3x_1^2 + a)^3}{(2y_1)^3} - y_1
\end{align*}

Эллиптическая кривая в короткой форме Вейерштрасса имеет следующий вид в проективных координатах:
\begin{equation*}
y^2z = x^3 + axz^2 + bz^3
\end{equation*}

Формулы сложения точек в проективных координатах~\cite{weierstrass-formulas}:
\begin{align*}
y_1z_1 &= y_1z_2 \\
x_1z_2 &= x_1z_2 \\
z_1z_2 &= z_1z_2 \\
u &= y_2z_1 - y_1z_2 \\
uu &= u \times u \\
v &= x_2z_1 - x_1z_2 \\
vv &= v \times v \\
vvv &= v \times vv \\
R &= vv \times x_1z_2 \\
A &= uu \times z_1z_2 - vv - 2R \\
x_3 &= v \times A \\
y_3 &= u(R - A) - vvv \times y_1z_2 \\
z_3 &= vvv \times z_1z_2
\end{align*}
В результате получается 7 сложений, 14 умножений.

Формулы удвоения точек~\cite{weierstrass-formulas}:
\begin{align*}
w &= a \times z_1^2 + 3 \times x_1^2 \\
s &= y_1 \times z_1 \\
ss &= s \times s \\
sss &= s \times ss \\
r &= y_1 \times s \\
b &= x_1 \times r \\
h &= w^2 - 8 \times b \\
x_3 &= 2 \times h \times s \\
y_3 &= w(4b - h) - 8 \times r^2 \\
z_3 &= 8 \times sss
\end{align*}
В результате получается 6 сложений, 16 умножений.

\subsection{Форма Монтгомери}

Кривая в форме Монтгомери определяется следующим сравнением~\cite{montgomery-curve}

$$
M_{\mu, \gamma}: \quad \gamma y_{m}^{2} \equiv x_{m}^{3}+\mu x_{m}^{2}+x_{m} \quad(\bmod p)
$$

Переход из формы Монтгомери в короткую форму Вейерштрасса задается одним из следующих способов.

Следуя (7), определим $\theta \equiv \frac{\gamma \mu}{3}(\bmod p)$. Тогда,

$$
\begin{aligned}
& a \equiv \gamma^{2}-3 \theta^{2} \equiv \gamma^{2}-\frac{\gamma^{2} \mu^{2}}{3} \equiv \gamma^{2}\left(1-\frac{\mu^{2}}{3}\right) \quad(\bmod p) \\
& b \equiv-\left(\theta^{3}+a \theta\right) \equiv-\theta\left(\gamma^{2}-2 \theta^{2}\right) \equiv \frac{2 \gamma^{3} \mu^{3}}{27}-\frac{\gamma^{3} \mu}{3} \equiv \frac{\gamma^{3} \mu}{3}\left(\frac{2 \mu^{2}}{9}-1\right) \quad(\bmod p)
\end{aligned}
$$

При этом, любая аффинная точка преобразуется следующим образом.

$$
\left(x_{m}, y_{m}\right) \rightarrow\left(x_{w} \equiv \gamma x_{m}+\frac{\gamma \mu}{3} \quad(\bmod p), \quad y_{w} \equiv \gamma^{2} y_{m} \quad(\bmod p)\right)
$$

Действительно, подставляя полученные равенства в уравнение кривой, получим

$$
\begin{aligned}
& x_{w}^{3}+a x_{w}+b \equiv\left(\gamma x_{m}+\frac{\gamma \mu}{3}\right)^{3}+\gamma^{2}\left(1-\frac{\mu^{2}}{3}\right)\left(\gamma x_{m}+\frac{\gamma \mu}{3}\right)+\frac{\gamma^{3} \mu}{3}\left(\frac{2 \mu^{2}}{9}-1\right) \equiv \\
& \equiv \gamma^{3}\left(x_{m}^{3}+x_{m}^{2} \mu+\frac{x_{m} \mu^{2}}{3}+\frac{\mu^{3}}{27}+x_{m}-\frac{x_{m} \mu^{2}}{3}+\frac{\mu}{3}-\frac{\mu^{3}}{9}+\frac{2 \mu^{3}}{27}-\frac{\mu}{3}\right) \equiv \\
& \equiv \gamma^{3}\left(x_{m}^{3}+x_{m}^{2} \mu+x_{m}\right) \equiv \gamma^{4} y_{m}^{2} \equiv y_{w}^{2} \quad(\bmod p) .
\end{aligned}
$$

Переход из формы Монтгомери в искривленную форму Эдвардса задается следующим образом. Если $\mu \not \equiv \pm 2(\bmod p)$ и $\gamma \not \equiv 0(\bmod p)$, то коэффициенты кривой определяются сравнениями

$$
e \equiv \frac{\mu+2}{\gamma} \quad(\bmod p), \quad d \equiv \frac{\mu-2}{\gamma} \quad(\bmod p)
$$

а преобразование отличных от нуля точек определяется следующим образом

$$
\left(x_{m}, y_{m}\right) \rightarrow\left(x_{e} \equiv \frac{x_{m}}{y_{m}} \quad(\bmod p), \quad y_{e} \equiv \frac{x_{m}-1}{x_{m}+1} \quad(\bmod p)\right)
$$

Действительно, подставляя полученные равенства в уравнение кривой в искривленной форме Эдвардса, получим

$$
\begin{gathered}
e x_{e}^{2}+y_{e}^{2} \equiv \frac{(\mu+2)}{\gamma} \frac{x_{m}^{2}}{y_{m}^{2}}+\frac{\left(x_{m}-1\right)^{2}}{\left(x_{m}+1\right)^{2}} \equiv \frac{(\mu+2) x_{m}^{2}}{x_{m}\left(x_{m}^{2}+\mu x_{m}+1\right)}+\frac{x_{m}^{2}-2 x_{m}+1}{x_{m}^{2}+2 x_{m}+1} \equiv \\
\equiv \frac{(\mu+2) x_{m}^{2}\left(x_{m}^{2}+2 x_{m}+1\right)+\left(x_{m}^{2}-2 x_{m}+1\right)\left(x_{m}^{3}+\mu x_{m}^{2}+x_{m}\right)}{\left(x_{m}^{3}+\mu x_{m}^{2}+x_{m}\right)\left(x_{m}^{2}+2 x_{m}+1\right)} \equiv ? \equiv \\
\equiv \frac{\left(x_{m}^{3}+\mu x_{m}^{2}+x_{m}\right)\left(x_{m}^{2}+2 x_{m}+1\right)+(\mu-2) x_{m}^{2}\left(x_{m}-1\right)^{2}}{\left(x_{m}^{3}+\mu x_{m}^{2}+x_{m}\right)\left(x_{m}^{2}+2 x_{m}+1\right)} \equiv 1+d x_{e}^{2} y_{e}^{2} \quad(\bmod p) .
\end{gathered}
$$

Групповой закон в аффинных координатах выглядит следующим образом:

Сложение точек
\begin{align*}
x_3 &= b \cdot \frac{(y_2 - y_1)^2}{(x_2 - x_1)^2} - a - x_1 - x_2 \\
y_3 &= \frac{(2 \cdot x_1 + x_2 + a) \cdot (y_2 - y_1)}{x_2 - x_1} - b \cdot \frac{(y_2 - y_1)^3}{(x_2 - x_1)^3} - y_1
\end{align*}

Удвоение точки
\begin{align*}
x_3 &= b \cdot \frac{(3 \cdot x_1^2 + 2 \cdot a \cdot x_1 + 1)^2}{(2 \cdot b \cdot y_1)^2} - a - x_1 - x_1 \\
y_3 &= \frac{(2 \cdot x_1 + x_1 + a) \cdot (3 \cdot x_1^2 + 2 \cdot a \cdot x_1 + 1)}{2 \cdot b \cdot y_1} - b \cdot \frac{(3 \cdot x_1^2 + 2 \cdot a \cdot x_1 + 1)^3}{(2 \cdot b \cdot y_1)^3} - y_1
\end{align*}

Эллиптическая кривая в форме Монтгомери имеет следующий вид в проективных координатах:
\begin{equation*}
a \cdot y^2z = x^3 + b \cdot x^2z + xz^2
\end{equation*}

\textbf{Формулы сложения точек:~\cite{montgomery-formulas}}
\begin{align*}
a &= x_1 - z_1 \\
b &= x_1 + z_1 \\
c &= x_2 - z_1 \\
d &= x_2 + z_2 \\
e &= a \cdot d \\
f &= b \cdot c \\
ef &= e + f \\
emf &= e - f \\
ef^2 &= ef \cdot ef \\
emf^2 &= emf \cdot emf \\
x_3 &= \text{minusz} \cdot ef^2 \\
z_3 &= \text{minusx} \cdot emf^2
\end{align*}
\textit{где minusz и minusx это подаваемые параметры, равны базовой точке во время лестницы Монтгомери. В результате получается 6 сложений, 6 умножений}

\textbf{Формулы удвоения точек:~\cite{montgomery-formulas}}
\begin{align*}
a &= x_1 + z_1 \\
b &= x_1 - z_1 \\
c &= a \cdot a \\
d &= b \cdot b \\
e &= c - d \\
x_3 &= c \cdot d \\
C &= \frac{(a + 2)}{4} \quad \text{(считается заранее)} \\
z_3 &= e \cdot (d + C \cdot e) 
\end{align*}
\textit{В результате получается 4 сложения, 5 умножений}



\subsection{Искривленная форма Эдвардса}
Кривая в искривленной форме Эдвардса определяется следующим сравнением ~\cite{edwards-curve}

$$
E_{e, d}: \quad e x_{e}^{2}+y_{e}^{2} \equiv 1+d x_{e}^{2} y_{e}^{2} \quad(\bmod p)
$$

Переход из искривленной формы Эдвардса в форму Монтгомери задается следующим образом. Рассматривая (10) как систему уравнений относительно неизвестных $\mu$ и $\gamma$, легко получить

$$
\begin{gathered}
\gamma(e+d)=2 \mu=>\mu=\frac{\gamma(e+d)}{2} \\
e=\frac{\mu+2}{\gamma}=\frac{\frac{\gamma(e+d)}{2}+2}{\gamma}=\frac{e+d}{4}+\frac{2}{\gamma}
\end{gathered}
$$

тогда

$$
\begin{gathered}
\frac{3 e-d}{4}=e-\frac{e+d}{4}=\frac{2}{\gamma}=>\gamma=\frac{8}{3 e-d} \\
\mu=\frac{\gamma(e+d)}{2}=\frac{4(e+d)}{3 e-d}
\end{gathered}
$$

$$
\begin{aligned}
& \mu \equiv \frac{2(e+d)}{e-d} \quad(\bmod p), \quad \gamma \equiv \frac{4}{e-d} \quad(\bmod p) \\
&\left(x_{e}, y_{e}\right) \rightarrow\left(x_{m} \equiv \frac{1+y_{e}}{1-y_{e}} \quad(\bmod p), \quad y_{m} \equiv \frac{x_{m}}{x_{e}} \quad(\bmod p)\right)
\end{aligned}
$$

Групповой закон в аффинных координатах

Сложение точек
\begin{align*}
x_3 &= \frac{x_1 \cdot y_2 + y_1 \cdot x_2}{1 + d \cdot x_1 \cdot x_2 \cdot y_1 \cdot y_2} \\
y_3 &= \frac{y_1 \cdot y_2 - a \cdot x_1 \cdot x_2}{1 - d \cdot x_1 \cdot x_2 \cdot y_1 \cdot y_2}
\end{align*}

Удвоение точки
\begin{align*}
x_3 &= \frac{x_1 \cdot y_1 + y_1 \cdot x_1}{1 + d \cdot x_1 \cdot x_1 \cdot y_1 \cdot y_1} \\
y_3 &= \frac{y_1 \cdot y_1 - a \cdot x_1 \cdot x_1}{1 - d \cdot x_1 \cdot x_1 \cdot y_1 \cdot y_1}
\end{align*}

Эллиптическая кривая в форме Скрученного Эдвардса имеет следующий вид в проективных координатах:
\begin{equation*}
a \cdot x^2z^2 + y^2z^2 = z^4 + d \cdot x^2 \cdot y^2
\end{equation*}

\textbf{Формулы сложения точек:~\cite{edwards-formulas}}
\begin{align*}
A &= z_1 \cdot z_2 \\
B &= A \cdot A \\
C &= x_1 \cdot x_2 \\
D &= y_1 \cdot y_2 \\
E &= d \cdot C \cdot D \\
F &= B - E \\
G &= B + E \\
x_3 &= A \cdot F \cdot ((x_1 + y_1) \cdot (x_2 + y_2) - C - D) \\
y_3 &= A \cdot G \cdot (D - a \cdot C) \\
z_3 &= F \cdot G
\end{align*}
\textit{В результате получается 6 сложений, 13 умножений}

\textbf{Формулы удвоения точек:~\cite{edwards-formulas}}
\begin{align*}
B &= (x_1 + y_1) \\
B &= B \cdot B \\
C &= x_1 \cdot x_1 \\
D &= y_1 \cdot y_1 \\
E &= a \cdot C \\
F &= E + D \\
H &= z_1 \cdot z_1 \\
J &= F - 2 \cdot H \\
x_3 &= (B - C - D) \cdot J \\
y_3 &= F \cdot (E - D) \\
z_3 &= F \cdot J
\end{align*}
\textit{В результате получается 6 сложений, 9 умножений}


\subsection{Форма Хадано}
Пусть $\gamma \in \mathbb{C}$ произвольный, отличный от нуля элемент, а $\theta$ - корень многочлена, стоящего в правой части равенства (2). Сделаем замену переменных

$$
u=\gamma(x-\theta), \quad v=\gamma^{\frac{3}{2}} y
$$

Тогда, записывая равенства

$$
x=\frac{u}{\gamma}+\theta, \quad y=\frac{1}{\gamma^{\frac{3}{2}}} v
$$

и подставляя их в (2), получим, что новые переменные удовлетворяют равенству

$$
\frac{1}{\gamma^{3}} v^{2}=\frac{u^{3}}{\gamma^{3}}+\frac{3 u^{2} \theta}{\gamma^{2}}+\frac{3 u \theta^{2}}{\gamma}+\theta^{3}+\frac{a u}{\gamma}+a \theta+b
$$

Сокращая на отличный от нуля элемент $\gamma^{-3}$ получим равенство

$$
v^{3}=u^{3}+3 \theta \gamma u^{2}+\left(3 \theta^{2}+a\right) \gamma^{2} u
$$

которое мы будем называть эллиптической кривой, заданной в форме Хадано~\cite{hadano-curve}.

Формулы удвоения точки
\begin{align*}
\lambda &= \frac{3x_1^2 + 2a_2x_1 + a_4}{2y_1} \\
x_3 &= \lambda^2 - a_2 - 2x_1 \\
y_3 &= (\lambda)(x_1 - x_3) - y_1
\end{align*}

Формулы сложения точек
\begin{align*}
\lambda &= \frac{y_2 - y_1}{x_2 - x_1} \\
x_3 &= \lambda^2 - a_2 - x_1 - x_2 \\
y_3 &= (\lambda)(x_1 - x_3) - y_1
\end{align*}

Алгоритм удвоения точек
\begin{align*}
y_1z_1 &= y_1 \times z_1 \\
y_1^2z_1 &= y_1z_1 \times y_1 \\
a_2z_1 &= a_2 \times z_1 \\
x_1^2 &= x_1 \times x_1 \\
a_2x_1^2z_1 &= a_2z_1 \times x_1^2 \\
z_1^2 &= z_1 \times z_1 \\
a_4z_1^2 &= a_4 \times z_1^2 \\
\lambda &= 3x_1^2 + a_2x_1^2z_1 + a_4z_1^2 \\
\lambda^2 &= \lambda \times \lambda \\
y_1^2z_1^4 &= 4 \times y_1^2z_1 \\
y_1^3z_1^2 &= y_1^2z_1^4 \times y_1z_1 \\
a_2z_1 + 2x_1 &= a_2z_1 + x_1 + x_1 \\
z_3 &= y_1^3z_1^2 \times (z_1 + z_1) \\
x_3 &= -y_1^2z_1^4 \times (a_2z_1 + 2x_1) + \lambda^2 \\
y_3 &= y_1^2z_1^4 \times (-(y_1^2z_1 + y_1^2z_1) + x_1 \times \lambda) - x_3 \times \lambda \\
x_3 &= x_3 \times (y_1z_1 + y_1z_1)
\end{align*}
В результате получили 10 сложений и 17 умножений.

Алгоритм сложения двух точек
\begin{align*}
z_1z_2 &= z_1 \times z_2 \\
y_2z_1 &= y_2 \times z_1 \\
y_1z_2 &= y_1 \times z_2 \\
x_2z_1 &= x_2 \times z_1 \\
x_1z_2 &= x_1 \times z_2 \\
\lambda &= y_2z_1 - y_1z_2 \\
\delta &= x_2z_1 - x_1z_2 \\
\lambda^2 &= \lambda \times \lambda \\
\delta^2 &= \delta \times \delta \\
\delta' &= x_2z_1 + x_1z_2 \\
a_2\delta^2 &= a_2 \times \delta^2 \\
x_3 &= z_1z_2 \times (\lambda^2 - a_2\delta^2) - \delta^2 \times \delta' \\
z_3 &= z_1z_2 \times \delta^2 \times \delta \\
y_3 &= z_2 \times \delta^2 \times (-y_1 \times \delta + \lambda \times x_1) - \lambda \times x_3 \\
x_3 &= x_3 \times \delta
\end{align*}
В результате получили 7 сложений и 18 умножений.
