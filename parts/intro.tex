Целью данной выпускной квалификационной работы является исследование эффективности использования комплексного умножения для вычисления кратных точек на эллиптических кривых. 
Исследование включает в себя построение эллиптической кривой с комплексным умножением, сравнение форм эллиптических кривых, анализ количества действий, а также разработку программной реализации.

Актуальность данного исследования обусловлена широким применением эллиптических кривых в современной криптографии, включая криптографические системы с открытым ключом. Эффективность форм эллиптических кривых для вычисления кратных точек напрямую влияет на скорость выполнения криптографических операций и, как следствие, на общую производительность криптографических систем.

Новизна данной работы заключается в отсутствии полноценного сравнения методов комплексного умножения с точки зрения количества необходимых операций для их выполнения, несмотря на растущий интерес к этой тематике в академическом сообществе. Дипломная работа состоит из введения, четырех основных глав, заключения, библиографического списка и приложения. В первой главе излагаются основные формы, используемые в работе. Вторая глава посвящена описанию эндоморфизма, используемого в работе. Третья глава фокусируется на определении основных алгоритмов вычисления кратной точки, используемых в вычислениях. Четвертая глава состоит из построения и сравнения результатов при вычислении кратной точки. В приложении представлен программный код, реализованный на языках программирования Python + Sage.
