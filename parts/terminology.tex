\begin{terminologyList}
    \term[Проективное пространство]{Пусть задано произвольное поле $F$. Множество векторов $P^2(F) = \{(x : y : z), x, y, z \in F\}$ определяем как двумерное проективное пространство над полем $F$. В данном пространстве вводится отношение эквивалентности: $(x : y : z) \sim (x_1 : y_1 : z_1)$, если существует $\lambda \in F$, $\lambda \neq 0$, такое что $x_1 = \lambda x$, $y_1 = \lambda y$, $z_1 = \lambda z$.~\cite{krypta-rozhok}}
    \term[Точка проективного пространства]{ Множество векторов, эквивалентных в рамках определенного выше отношения, представляет собой точку введенного проективного пространства.~\cite{krypta-rozhok}}
    \term[Характеристика поля]{ Характеристикой поля $F$ является наименьшее неотрицательное целое число $p$, для которого выполняется равенство $p \cdot a = 0$ для любого элемента $a \in F$.~\cite{krypta-rozhok}}
    \term[Эллиптическая кривая]{ При характеристике поля $F$ больше 3, рассмотрим однородное уравнение эллиптической кривой вида $y^2z = x^3 + axz^2 + bz^3$ (уравнение 1), где $4a^3 + 27b^2 \neq 0$, $a, b \in F$. Множество точек проективного пространства $P^2(F)$, удовлетворяющих данному уравнению, называется эллиптической кривой над полем $F$.~\cite{krypta-rozhok}}
    \term[Бесконечно удаленная точка]{ Бесконечно удаленная точка эллиптической кривой (обозначается $O$) определяется как точка, векторы которой эквивалентны вектору $(0 : 1 : 0)$. Для каждого такого вектора уравнение (1) выполняется, и последняя координата равна 0.~\cite{krypta-rozhok}}
    \term[Эллиптической кривая в аффинной форме]{ Множество решений уравнения (1) в аффинной форме вместе с бесконечно удаленной точкой $O$ образует эллиптическую кривую над полем $F$. Аффинная точка эллиптической кривой представляет собой пару $(x, y)$, удовлетворяющую уравнению $y^2 = x^3 + ax + b$ (уравнение 2).~\cite{krypta-rozhok}}
    \term[Аддитивная Абелева группа]{ Аддитивная Абелева группа – это такая группа, где групповая операция (в данном случае сложение) является коммутативной. На множестве точек эллиптической кривой можно определить структуру абелевой группы.~\cite{krypta-rozhok}}
    \term[Обратный элемент]{ Для любой точки эллиптической кривой $(x_1 : y_1 : z_1)$ существует обратная точка $(x_1 : -y_1 : z_1)$.~\cite{krypta-rozhok}}
    \term[Скалярное умножение]{ Скалярное умножение точки $P$ эллиптической кривой на натуральное число $n$ определяется как $[n]P = P + P + \cdots + P$ ($n$ раз).~\cite{krypta-rozhok}}
    \term[Конечные поля и порядок группы]{ Если поле $F$ конечно, то группа точек эллиптической кривой также конечна. Порядок группы точек эллиптической кривой – это количество элементов такой группы, обозначается $|E(F)|$.~\cite{krypta-rozhok}}
    \term[Уравнения над кольцом вычетов]{ Уравнения эллиптических кривых можно рассматривать над произвольным кольцом вычетов $Z_n$. В этом случае уравнение принимает вид: $y^2z \equiv x^3 + axz^2 + bz^3 \pmod{n}$, $a, b, x, y, z \in Z_n$.~\cite{krypta-rozhok}}
    \term[Эндоморфизм]{ Эндоморфизмом эллиптической кривой называется такое гомоморфное отображение кривой в себя, которое сохраняет структуру группы. То есть, для эндоморфизма $\phi$ и любых точек $P, Q$ на эллиптической кривой $E$, выполняется условие $\phi(P + Q) = \phi(P) + \phi(Q)$.~\cite{nesterenko-disser}}
    \term[Комплексное умножение]{ Комплексное умножение – это класс эллиптических кривых, для которых существует не только эндоморфизм умножения на число, но и эндоморфизмы, соответствующие умножению на некоторые комплексные числа. Это расширение классической концепции умножения точки кривой на целое число.~\cite{nesterenko-disser}}
    \term[Изоморфизм эллиптических кривых]{ Изоморфизмом между двумя эллиптическими кривыми $E_1$ и $E_2$ называется биективное гомоморфное отображение $\phi: E_1 \to E_2$, которое сохраняет структуру группы. Другими словами, $\phi$ сопоставляет каждой точке на $E_1$ уникальную точку на $E_2$ таким образом, что сложение точек сохраняется.~\cite{krypta-rozhok}}
    \term[Торсионные точки]{ Торсионными точками эллиптической кривой называются такие точки, которые при умножении на некоторое натуральное число $n$ дают нулевую точку (точку на бесконечности) кривой. Множество всех торсионных точек кривой образует торсионную подгруппу.~\cite{krypta-rozhok}}
    \term[Конечное поле]{ Конечным полем называется поле, содержащее конечное количество элементов. Эллиптические кривые над конечными полями обладают рядом уникальных свойств, делающих их полезными в криптографии.~\cite{krypta-rozhok}}
    \term[Ранг эллиптической кривой]{ Ранг эллиптической кривой над конечным полем определяется как максимальное число линейно независимых точек бесконечного порядка на этой кривой. Ранг является важным параметром при анализе структуры группы точек на кривой.~\cite{nesterenko-disser}}
\end{terminologyList}
