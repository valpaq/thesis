Пусть $d \in \mathbb{N}$ свободное от квадратов число, $\tau \in \mathbb{Q}(\sqrt{-d})$ - мнимая квадратичная иррациональность, принадлежащая верхней комплексной полуплоскости, и $\Lambda_{\tau}=\{n+m \tau, n, m \in \mathbb{Z}\}$ - решетка в $\mathbb{C}$. 
Известно, см. ~\cite{endo-reshetka}, что существует только 11 решеток $\Lambda_{\tau}$ с числом классов эквивалентных квадратичных форм, равным единице, и значением модулярной функции $j(\tau)$, отличным от нуля и 1728. 
Каждой такой решетке соответствует одна, с точностью до изоморфизма, эллиптическая кривая вида(2) с целым значением инварианта $j(E)=1728 \frac{4 a^{3}}{4 a^{3}+27 b^{2}} \in \mathbb{Z}$ и кольцом эндоморфизмов, изоморфным $\Lambda_{\tau}$. 
Для пяти из рассматриваемых кривых существует рациональный корень $\theta$ многочлена. 
Перечень таких кривых приводится в таблице 2, третий столбец таблицы содержит значение нормы величины $\tau$, а в четвертый столбец - точное значение инварианта эллиптической кривой~\cite{nesterenko-disser}.

\begin{table}[h]
    \centering
    \begin{tabular}{|r|r|c|r|l|c|}
    \hline
    № & $\tau$ & $N(\tau)$ & $j(E)$ & $E_{\tau}$ & $\theta$ \\
    \hline
    1 & $2 \sqrt{-1}$ & 4 & 287496 & $y^{2}=x^{3}-\frac{11}{4} x-\frac{7}{4}$ & -1 \\
    \hline
    2 & $\sqrt{-2}$ & 2 & 8000 & $y^{2}=x^{3}-\frac{15}{2} x-7$ & -2 \\
    \hline
    3 & $\sqrt{-3}$ & 3 & 54000 & $y^{2}=x^{3}-\frac{15}{4} x-\frac{11}{4}$ & -1 \\
    \hline
    5 & $\frac{1}{2}(1+\sqrt{-7})$ & 2 & -3375 & $y^{2}=x^{3}-\frac{35}{4} x-\frac{49}{4}$ & $\frac{7}{2}$ \\
    \hline
    6 & $\sqrt{-7}$ & 7 & 16581375 & $y^{2}=x^{3}-\frac{595}{4} x-\frac{2793}{4}$ & -7 \\
    \hline
    \end{tabular}
    \caption{ Эллиптические кривые с целым $j$-инвариантом и рациональным корнем $\theta$ Полный список из 11 кривых предоставлен в работе ~\cite{nesterenko-disser}}
\end{table}

Наличие рационального корня $\theta$ позволяет построить отображения перечисленных эллиптических кривых в различные формы, и провести сравнение реализаций.

В данной работе рассмотрена эллиптическая кривая №3 в короткой форме Вейерштрасса:
\begin{equation*}
    y^2 = x^3 - \frac{15}{4}x - \frac{11}{4},
\end{equation*}
с параметрами \( \theta = -1 \) и \( \delta = \sqrt{-3} \).

Форма Хадано для этой кривой при \( \beta = 2 \) выражается как:
\begin{equation*}
    v^2 = u^3 - 6u^2 - 3u.
\end{equation*}

Эндоморфизм кривой задаётся следующим образом:
\begin{equation*}
    \phi:(u,v) = \left(-\frac{u(u-3)^2}{3(u+1)^2}, -\frac{(u^3+3u^2-21u+9)v}{3\alpha(u+1)^3}\right).
\end{equation*}

Алгоритм для проективных координат представляется следующим образом:
\begin{align*}
    xx1 &= x + z, \\
    xx1xx1 &= xx1 \cdot xx1, \\
    z_{13} &= 3 \cdot z, \\
    xx3 &= x - z_{13}, \\
    \delta &= \alpha \cdot xx1, \\
    z_{\text{new}} &= z_{13} \cdot \delta \cdot xx1xx1, \\
    x_{\text{new}} &= -\delta \cdot x \cdot xx3^2, \\
    x1z_{13} &= x \cdot z_{13}, \\
    y_{\text{new}} &= xx3 \cdot (x^2 + x1z_{13} + x1z_{13} - z_{13} \cdot z) \cdot y.
\end{align*}
Получается 5 сложений и 13 умножений.
