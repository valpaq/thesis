\subsection*{Алгоритм add-and-double}
Основным алгоритмом, используемым для получения кратной точки, является \textit{add-and-double}. Этот алгоритм лежит в основе многих криптографических операций, таких как генерация ключей и создание цифровых подписей. Он применяет основные операции с эллиптическими кривыми: сложение и удвоение точек.

Алгоритм \textit{add-and-double} выполняется следующим образом:

\begin{enumerate}
    \item Имеется точка $P$ на эллиптической кривой $E$, которую нужно умножить на число $k$, чтобы получить $kP$.
    \item Число $k$ представляется в бинарном виде. Например, для $k = 12$ бинарное представление будет $1100$.
    \item Процесс итерации по битам числа $k$ (1 и 0):
    \begin{itemize}
        \item Инициализируется точка на бесконечности $N$.
        \item Для каждого бита, начиная со старшего, происходит удвоение текущей точки $N$. Если бит равен 1, к результату добавляется точка $P$.
    \end{itemize}
\end{enumerate}

Данный алгоритм эффективен для вычисления кратных точек на эллиптических кривых, сокращая количество необходимых операций по сравнению с простым последовательным сложением.

Алгоритм \textit{add-and-double} не подходит для задачи вычисления кратной точки с эндоморфизмом в виде комплексного умножения по двум причинам:

\begin{itemize}
    \item Комплексные числа не могут быть представлены в бинарном виде.
    \item В алгоритме не используется эндоморфизм.
\end{itemize}


\subsection*{Алгоритм add-and-map}

В работе ~\cite{nesterenko-disser} предложен алгоритм, позволяющий представить число в виде многочлена из комплексного числа.

\begin{enumerate}
    \item Задается элемент $a$, для которого определены $tr(a) = a + \bar{a}$ и $N(a) = a \cdot \bar{a}$.
    \item Создается массив $wArray$, ограниченный сверху значением $\text{Round}[2 \cdot \log N(a) \cdot k] + 4$ (согласно Теореме 1.6 ~\cite{nesterenko-disser}).
    \item Устанавливаются начальные значения: $s_0 = k$, $s_1 = 0$, $w = 0$, $\theta(a) = a \mod 2$, $nAa = (N(a) - \theta(a)) / 2$.
    \item Итерация по массиву:
    \begin{itemize}
        \item Вычисляется $x_w = s_0 \mod nAa$.
        \item Если $x_w > nAa$, то корректируется $x_w = x_w - N(a)$.
        \item Увеличивается $q$ на 1.
        \item В массив $wArray$ на позицию $w$ записывается значение $x_w$.
        \item Обновляются $s_0 = q \cdot tr(a) + s_1$ и $s_1 = -q$.
        \item Инкременируем $w$.
    \end{itemize}
\item Алгоритм завершается, когда выполняются условия: $|s_0| < nAa$ и $s_1 = 0$.
\end{enumerate}



Для сборки полученного массива для получения числа используется следующий алгоритм:

\begin{enumerate}
\item Исходная точка обозначается как $P$, а вторая точка $Q$ инициализируется как $O$.
\item Итерация по каждому элементу массива:
\begin{itemize}
\item Если элемент массива $\neq 0$, то $Q = Q + \text{element} \cdot R$.
\item Происходит обновление $Q = \phi(Q)$.
\end{itemize}
\item В результате получается последовательность, аналогичная той, что была использована для разбиения скаляра, только теперь она собрана в одно целое.
\end{enumerate}

При этом эти алгоритмы не подходят для эллиптической кривой в форме Монтгомери с XZ координатами.


\subsection*{Алгоритм Montgomery Ladder}

В 1987 году американский математик Питер Монтгомери~\cite{montgomery-curve} предложил алгоритм, который устраняет необходимость в использовании $y$-координаты для вычисления скалярного произведения точки на эллиптической кривой, при этом обеспечивая дополнительную защиту от атак, связанных с анализом энергопотребления.

Алгоритм "Лестницы Монтгомери" представляет собой метод ускоренного скалярного умножения точки $P$ на эллиптической кривой по скаляру $k$. Он использует две временные переменные, $Q_0$ и $Q_1$. 


\begin{enumerate}
    \item Инициализация переменных: Устанавливаются начальные значения, включая $Q_0 = P$ и $Q_1 = 2P$.
    \item Итерации: Цикл выполняется для каждого бита скаляра $k$ (начиная с самого старшего бита).
        \begin{itemize}
            \item Если бит равен 0, выполняется операция удвоения точки $Q_0$ и сложения точки $Q_1$.
            \item Если бит равен 1, выполняется операция удвоения точки $Q_1$ и сложения точки $Q_0$.
        \end{itemize}
\end{enumerate}

Алгоритм "Лестницы Монтгомери" минимизирует количество операций умножения, что делает его эффективным для криптографических операций на эллиптических кривых, таких как создание открытых ключей и цифровые подписи.

Подсчет $Q_0$ и $Q_1$ на каждом шаге алгоритма является необходимым условием. Поскольку Монтгомери отказывается от использования $y$-координаты[9], необходимо иметь точный порядок действий для вычисления проективных координат на каждой итерации.