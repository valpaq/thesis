\documentclass[14pt]{HSEUniversity}
\usepackage{graphicx}
\addbibresource{parts/refs.bib}
\begin{thesis}[
    title = Построение эллиптических кривых с комплексным умножением для криптографических приложений,
    year = 2024,
    %
    authorGroup = СКБ182,
    authorName = В. А. Парамонов,
    %
    academicTeacherTitle = {Доцент кафедры Компьютерной безопасности НИУ ВШЭ МИЭМ},
    academicTeacherName = А.Ю. Нестеренко.,
    %
    keywordsRu = эллиптическая кривая; вычисление кратной точки; эндоморфизм; комплексное умножение,
    keywordsEn = elliptic curve; point multiplication; endomorphism; complex multiplication;
]

\setAbstractResource{parts/abstract-ru}{parts/abstract-en}
\setTerminologyResource{parts/terminology}
\setIntroResource{parts/intro}
\addChapter{Сравниваемые формы эллиптических кривых}{parts/forms-and-connections}
\addChapter{Эндоморфизм эллиптической кривой}{parts/endomorphism}
\addChapter{Алгоритмы возведения в кратную точку}{parts/multiple-point-info}
\addChapter{Построение эллиптической кривой и экспериментальные вычисления}{parts/analysis}
\setConclusionResource{parts/conclusion}

\addAppendix{Построение эллиптической кривой}{parts/constructing-elliptic}
\addAppendix{Формулы сложения и удвоения точек в разных формах}{parts/formulas-for-points-addition}
\addAppendix{Формулы сложения и удвоения точек в разных формах}{parts/formulas-for-points-addition}
\addAppendix{Изогении между формами кривых}{parts/isogenii}
\addAppendix{Алгоритмы вычисления кратной точки}{parts/multiple-point}
\addAppendix{Построение граффиков}{parts/python-graphics}



\end{thesis}
